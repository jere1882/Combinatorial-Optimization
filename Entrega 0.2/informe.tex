\documentclass {article} 

\usepackage{lmodern}
\usepackage [spanish] {babel} 
\usepackage [T1]{fontenc}
\usepackage [latin1]{inputenc}
\usepackage{amsthm} % para poder usar newtheorem
\usepackage{cancel} %Para poder hacer el simbolo "no es consecuencia sem�ntica" etc.
\usepackage{graphicx} 
\usepackage{amsmath} %para poder usar mathbb
\usepackage{amsfonts} %sigo intentando usar mathbb
\usepackage{amssymb} %therefore
\usepackage{ mathabx } %comillas
\usepackage{ verbatim } 
\theoremstyle{remark}
\newtheorem{thm}{Teorema}
\newtheorem{lem}{Lema}[section]
\newtheorem{cor}{Corolario}[section]
\newtheorem{deff}{Definici�n}[section]
\newtheorem{obs}{Observaci�n}[section]
\newtheorem{ej}{Ejercicio}[section]
\newtheorem{ex}{Ejemplo}[section]
\newtheorem{alg}{Algoritmo}[section]
\usepackage[latin1]{inputenc} 
\usepackage{listings}
\usepackage{verbatim}
\usepackage{hyperref}
\usepackage{proof}
\usepackage{anysize}
\marginsize{3cm}{3cm}{3cm}{3cm}
\usepackage{tikz}
\usepackage{ stmaryrd }

\begin{document} 



\begin{titlepage}

\newcommand{\HRule}{\rule{\linewidth}{0.5mm}} % Defines a new command for the horizontal lines, change thickness here

\center % Center everything on the page
 
%----------------------------------------------------------------------------------------
%    HEADING SECTIONS
%----------------------------------------------------------------------------------------

\textsc{\Large Facultad de Ciencias Exactas, Ingenier�a y Agrimensura}\\[1.5cm] % Name of your university/college

\textsc{ T�picos Avanzados en Optimizaci�n Combinatoria y Teor�a de Grafos}\\[0.5cm] % Minor heading such as course title

%----------------------------------------------------------------------------------------
%    TITLE SECTION
%----------------------------------------------------------------------------------------

\HRule \\[0.4cm]
{ \huge \bfseries PR�CTICA 0.2}\\[0.4cm] % Title of your document
\HRule \\[1.5cm]
 
%----------------------------------------------------------------------------------------
%    AUTHOR SECTION
%----------------------------------------------------------------------------------------

%\begin{minipage}{0.4\textwidth}
%\begin{flushleft} \large
%\emph{Autor:}\\
%Rodr�guez Jerem�as % Your name
%\end{flushleft}
%\end{minipage}

%\begin{minipage}{0.4\textwidth}
%\begin{flushright} \large
%\emph{Profesor:} \\
%Mauro Jaskelioff % Supervisor's Name
%\end{flushright}
%\end{minipage}\\[4cm]

% If you don't want a supervisor, uncomment the two lines below and remove the section above
\Large \emph{Alumno: Rodr�guez Jerem�as}\\


%----------------------------------------------------------------------------------------
%    DATE SECTION
%----------------------------------------------------------------------------------------

{\large \today}\\[3cm] % Date, change the \today to a set date if you want to be precise

%----------------------------------------------------------------------------------------
%    LOGO SECTION
%----------------------------------------------------------------------------------------

%\includegraphics{Logo}\\[1cm] % Include a department/university logo - this will require the graphicx package
 
%----------------------------------------------------------------------------------------

\vfill % Fill the rest of the page with whitespace

\end{titlepage}
\section{Ejercicio 3A} 
Dados los par�metros $A \in \mathbb{R}^{m \times n}; \ b, c \in \mathbb{R}^n$ y $x^* \in \{0,1\}^n$ del problema, realizamos dos formulaciones en programaci�n lineal entera:

\begin{flushleft}
 \underline{Primer formulaci�n:}
\end{flushleft}


\begin{eqnarray}
\nonumber && min \ c x \\
\nonumber &s/a&\\
\nonumber & &  Ax \leq b  \\
\nonumber & & x_i \in \mathbb{Z}, \ \ \ \ \ \ \ \ \forall \ i \in \{1, \ldots, n\} \\
\nonumber & & 0 \leq x_i \leq q  \ \ \ \ \forall \ i \in \{1, \ldots, n\} \\
\nonumber & &\sum_{i=1}^n ( (x_i . x_i^*)+ ( (1-x_i).(1-x_i^*)) ) < n
\end{eqnarray}


En esta �ltima desigualdad, cada t�rmino de la sumatoria vale 1 si la componente i-�sima de $x$ es igual a la de $x^*$; y 0 en caso contrario. De este modo, la �ltima restricci�n impone que no todas las componentes sean iguales; es decir, que $x \neq x^*$.

\begin{flushleft}
 \underline{Segunda formulaci�n:}
\end{flushleft}


\section{Ejercicio 7}
Claramente:
\begin{equation*}  
dom(\Pi)=\{(0,0,0,0),(0,0,0,1),(0,0,1,0),(0,0,1,1),(0,1,0,0),(0,1,0,1),(1,0,0,0)\} =  dom(\Pi')
\end{equation*}
\par Adem�s, $\Pi$ y $ \Pi'$ tienen la misma funci�n objetivo. Por lo tanto, son problemas equivalentes. \\
\par Sea $P$ el poliedro dominio de la relajaci�n lineal de $\Pi$, y $P'$ el de $\Pi'$. Para probar que $\Pi'$ es una mejor formulaci�n que $\Pi$, veamos que $P' \subset P$.\\
\par Sea $(x_1,x_2,x_3,x_4) \in P'$. Luego, $4x_1+3x_2+2x_3+x_4 \leq 4$. Equivalentemente, $100 x_1+75 x_2+50 x_3+25 x_4 \leq 100$, de donde se deduce $83 x_1+61 x_2+49 x_3+20 x_4 \leq 100$. 
\par Adem�s, $(\frac{6}{5} ,0,0,0)\in (P - P')$.

\begin{flushright} $\therefore P' \subset P$ \end{flushright}

\newpage
\section{Ejercicio 8.a}

Realizamos la siguiente formulaci�n del problema en formato lp:

\begin{verbatim}
/* Variables */

// xj : cantidad de jamones no ahumados producidos
// xl : cantidad de lomos no ahumados producidos
// xs : cantidad de salchichas no ahumadas producidas

// xja: cantidad de jamones ahumados producidos (sin emplear horas extra)
// xla: cantidad de lomos ahumados producidos (sin emplear horas extra)
// xsa: cantidad de salchichas ahumadas producidas (sin emplear horas extra)

// xje: cantidad de jamones ahumados en horas extra
// xle: cantidad de lomos ahumados en horas extra
// xse: cantidad de salchichas ahumadas en horas extra

// j : se enciende la m�quina J (variable de decisi�n, 0-1)
// l : se enciende la m�quina L (variable de decisi�n, 0-1)
// s : se enciende la m�quina S (variable de decisi�n, 0-1)

/* Objective function */
MAX: 8 xj + 4 xl + 4 xs + 14 xja + 12 xla + 13 xsa + 11 xje + 7 xle + 9 xse - 50j - 50l - 50s;

/* Variable bounds */

xj + xja + xje - 10 j <= 480;  // Producci�n de jamones
xl + xla + xle - 40 l <= 400;  // producci�n de lomos
xs + xsa + xse - 20 s <= 230;  // producci�n de salchichas

xja + xla + xsa <= 420;       // capacidad de ahumado
xje + xle + xse <= 250;       // capacidad de ahumado extra

xj  >=0;
xja >=0;
xje >=0;
xl  >=0;
xla >=0;
xle >=0;
xs  >=0;
xsa >=0;
xse >=0;

0 <= j <= 1;
0 <= s <= 1;
0 <= l <= 1;
                     //         J L S
j + l + s <= 2;      // evita   1 1 1
j <= l + s;          // evita   1 0 0
l <= j + s;          // evita   0 1 0

int j;
int s;
int l;
\end{verbatim}

Usando LPSolve, obtuvimos que el valor m�ximo de la funci�n objetivo es $11110$ para la siguiente asignaci�n de valores:

\begin{verbatim}
xj = 480
xl = 20
xs = 0
xja = 0
xla = 420
xsa = 0
xje = 0
xle = 0
xse = 250
j = 0
l = 1
s = 1
\end{verbatim}



\end{document}
